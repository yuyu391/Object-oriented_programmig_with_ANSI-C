% # Copyright (C) 2009-2014 the Fandol Team, All wrongs reserved.
% # -*- coding: utf-8 -*-
% !TEX encoding = UTF-8 Unicode

% Version Control System Information: Subversion, host on Google Code;
% FileID:		$Id$;
% FileDate:		$Date$;
% FileRevision:	$Revision$

% \chapter{Programming Savvy ---Arithmetic Expressions}
\chapter{编程常识:算数表达式}
\label{ch:psArithmeticExpressions}

动态链接就其本身而言是一种很有用的程序规划技术。
相比写比较少的附带许多 \ccode{switch} 语句控制许多特殊情况的函数,
不如写许多短的函数,这样,在每种情况下,都能很好地动态链接。
这也常常简化了我们的程序工作并且它常常使我们的代码容易扩展。


作为例子,我们将要写一个计算由浮点数、圆括号
及一些常用的加、减等算术符号组成的算术表达式的程序。
一般说来,我们将要通过规定的编译工具和另外的某编译器
来生成能计算这个算术运算的目标程序。
当然,这本书不是谈论编译器的工作原理,
所以,这一次我们将亲自写出程序代码.

% \section{The Main Loop}
\section{主循环}
\label{sec:mainloop}

这个程序的主循环先从标准输入中读取一行,
并且根据它进行部分初始化,
这样,数字和操作符就能被提取出来,而且空格也被忽略掉了。
然后调用一个函数来识别这个正确的算术表达式,
并以某种方式存储它,最后处理这些存储的结果。
如果出错,那我们可以简单地阅读下一个输入行。下面是主函数:
\begin{lstlisting}
#include <setjmp.h>
static enum tokens token; /* current input symbol */ /*当前输入信号*/
static jmp_buf onError;
int main (void)
{   
    volatile int errors = 0;
    char buf [BUFSIZ];

    if (setjmp(onError))
    ++ errors;
    while (gets(buf))
        if (scan(buf))
        {   void * e = sum();
            if (token)
            error("trash after sum");
            process(e);
            delete(e);
        }
    return errors > 0;
}
void error (const char * fmt, ...)
{   
    va_list ap;

    va_start(ap, fmt);
    vfprintf(stderr, fmt, ap), putc(’\n’, stderr);
    va_end(ap);
    longjmp(onError, 1);
}
\end{lstlisting}

错误校正在 \ccode{setjmp()} 中定义了。
如果 \ccode{error()} 函数在程序的某个地方被调用,
则伴随着 \ccode{setjmp()} 的另一个循环,\ccode{longjmp()} 将继续被执行。
在这种情况下,结果就就传给了 \ccode{longjmp()},错误也被保存了,
同时下一个输入行仍有效。
离开当前程序时将报告这段代码是否有错误。


% \section{The Scanner}
\section{扫描器}
\label{sec:scanner}

在主循环中,一旦输入数据被传入 \ccode{buf[]},
它就会传递至 \ccode{scan()} 函数,
为每一处调用函数的地方提供下一个输入变量标志。
在每行的结束是零标记。

\begin{lstlisting}
#include <ctype.h>
#include <errno.h>
#include <stdlib.h>
#include "parse.h" /* defines NUMBER */
static double number; /* if NUMBER: numerical value */
static enum tokens scan (const char * buf)
/* return token = next input symbol */
{   
    static const char * bp;
    if (buf)
        bp = buf; /* new input line */
    while (isspace(* bp))
        ++ bp;
    if (isdigit(* bp) || * bp == ’.’)
    {   errno = 0;
        token = NUMBER, number = strtod(bp, (char **) & bp);
        if (errno == ERANGE)
            error("bad value: %s", strerror(errno));
    }
    else
        token = * bp ? * bp ++ : 0;
    return token;
}
\end{lstlisting}

我们通过输入的地址或者空指针调用scan()函数来继续运行。
为了求出符合ANSI-C 标准的strtod()函数中的浮点数,
我们忽略了空格,和首要的数字或小数点。其它的任何特性仍不变,% TODO
并且我们不会在输入的缓冲提前输出以前的空位。


% \section{The Recognizer}
\section{识别器}
\label{sec:recogn}

\ccode{scan()} 函数的结果被存取在全局变量中
——这简化了识别器的工作。
如果我们发现了一个数值,
我们就返回它的唯一值并且确保它的全局变量有效。

% \section{The Processor}
\section{The Processor}<++>

% \section{Information Hiding}
\section{信息隐藏}<++>

% \section{Dynamic Linkage}
\section{Dynamic Linkage}<++>

% \section{A Postfix Writer}
\section{A Postfix Writer}<++>

% \section{Arithmetic}
\section{Arithmetic}<++>

% \section{Infix Output}
\section{Infix Output}<++>

% \section{Summary}
\section{小结}


\newpage{\thispagestyle{empty}\cleardoublepage}
% vim: set syntax=tex ts=4 sw=4 tw=76 fo+=Mm cc=+2 noundofile nobackup :

