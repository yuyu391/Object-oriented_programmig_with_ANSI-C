% # Copyright (C) 2009-2014 the Fandol Team, All wrongs reserved.
% # -*- coding: utf-8 -*-
% !TEX encoding = UTF-8 Unicode

% Version Control System Information: Subversion, host on Google Code;
% FileID:		$Id$;
% FileDate:		$Date$;
% FileRevision:	$Revision$

% \chapter{Programming Savvy ---Symbol Table}
\chapter{编程常识:符号表}
\label{ch:psSymbolTable}
    正确的扩展结构体,并且,分享基础结构体的函数,能帮助我们避免用union的麻烦.尤其是在
结合动态连接,我们获得一个制度和完美稳健处理发散信息的方法.一旦基础的机制确定好位置,
那么新的扩展结构体能很容易的添加到基础机制中并且重复使用.
    
    作为一个示例,我们将添加关键字,常量,变量,数学表达式到第三章的小计算器中.所有添加
的项目都生存字符号表中并且分享同样的基础名字的搜索机制.

% \section{Scanning Identifiers}
\section{标识符扫描}
    在3.2章节中,我们实现了\ccode{scan()}函数,其能从主进程接收每一行输入并且在每一次
搜寻中维持一个输入符号.如果我们想要介绍关键字,常量等等,我们需要扩展下\ccode{scan()}.
就像浮点数一样,我们提取字母字符串最为进一步的分析:
\begin{lstlisting}
#define ALNUM   "ABCDEFGHIJKLMNOPQRSTUVWXYZ" \
                "abcdefghijklmnopqrstuvwxyz" \
                "_" "0123456789"
static enum tokens scan (const char * buf)
{
    static const char * bp;
    if (isdigit(* bp || * bp == '.'))
        ...
    else if (isalpha(* bp) * bp == '_')
    {
        char buf [BUFSIZ];
        int len = strspn(bp, ALNUM);
        
        if (len >= BUFSIZ)
            error("name too long: %—.10s...", bp); 
        strncpy(buf, bp, len), buf[len] = ’\0’, bp += len; 
        token = screen(buf);
    }
    ...
}
\end{lstlisting}
    
    一旦我们有一个标识符,我们让\ccode{screen()}一个新的函数来决定\ccode{token}的
取值.如果必要的,\ccode{screen()}函数会存储一个描述符到解析器可以检查的全局变量描
述符中.

% \section{Using Variables}
\section{使用变量}
    一个变量参与两个操作:它的值被用作于操作数在一个表达式中,或者一个表达式的值被
分配给它.第一个操作是一个简单\ccode{factor()}函数的延伸,其在3.5章节中作为识别器
的一部分.

\begin{lstlisting}
static void * factor (void)
{
    void * result:
    ...
    switch (token) {
    case VAR:
        result = symbol;
        break;
    ...
    }
    ...
}
\end{lstlisting}

    VAR是一个特殊的值,当一个合适的描述符被发现时\ccode{screen()}替代token的值.额外
的关于描述符的信息是被安排于全局变量符号中.在这个例子中,符号包含一个代表作为表达式
树的叶子变量的节点.\ccode{screen()}函数能找到任何一个在描述符表中的变量或者用描述
符Var去创建一个.

    识别某一个分配是一个bit变得更加复杂.我们的计算器是一个非常实用的,我们允许两
种语法声明:
\begin{lstlisting}
asgn : sum
    | VAR = asgn
\end{lstlisting}
不幸的是,VAR也能出现也sum表达式的左边,i.e.,它不能立即清楚的如何认出c-style嵌入式
分配用我们的递归下降技术.这是因为我们想要学习如何去处理关键字不论用任何方法,我们
设定遵循这样的语法规则:
\begin{lstlisting}
stmt : sum
    | LET  VAR = sum
\end{lstlisting}
下面是遵循这一规则的函数实现:
\begin{lstlisting}
static void * stmt (void) 
{ 
    void * result; 
    switch (token) 
    { 
    case LET: 
        if (scan(0) != VAR) 
            error("bad assignment"); 
        result = symbol; 
        if (scan(0) != ’=’) 
            error("expecting ="); 
        scan(0); 
        return new(Assign, result, sum()); 
    default: 
        return sum(); 
    } 
}
\end{lstlisting}

    在中程序中,我们用\ccode{stmt()}替代\ccode{sum()}并且我们的识别器准备捕捉变量.
\ccode{Assign}是一个新的数据类型描述符,其作为一个计算值的节点和分配值到一个变量.


% \section{The Screener ---\cemph{Name}}
\section{The Screener ---\cemph{Name}}
分配器遵循这样的语法:
\begin{lstlisting}
stmt : sum
    | LET VAR = sum
\end{lstlisting}
LET 是一个关键字的例子.在building中 the screener 我们仍然能决定什么标识符将是当
时的代表LET:\ccode{scan()}提取出一个标识符从输入行里,并把它压进\ccode{screen()}
中,\ccode{screen()}看起来像在符号表中其返回值作为token,至少是一个变量,一个符号
节点.

    识别器丢弃LET但是它安装变量作为一个叶子节点到树上.对于其他符号,比如算术表达
式的名字,我们将希望应用\ccode{new()}不论screener返回什么样的符号而得到一个树上的
新节点.因此,有很多部分组成的符号表条目将会有一样动态连接树节点的函数.

    作为关键字,一个\ccode{Name}需要包含输入字符串和token值.此后,我们想要从
\ccode{Name}继承;因此,我们定义一个这样的结构体在Name.r文件中:
\begin{lstlisting}
struct Name {               /* base structure */
    const void * type;      /* for dynamic linkage */
    const char * name;      /* may be malloc-ed */
    int token;
};
\end{lstlisting}
我们的符号永不消逝:如果他们的名字包含作为预定义的关键字的字符串,那也没有关系,或
者作为用户定义的变量的动态存储字符串--我们不会回收他们.

    在我们能找到一个符号之前,我们需要把它输入符号表.这个不能通过调用
\ccode{new(Name,...)}来处理,因为我们希望支持比Name更加复杂化的符号,我们应该隐藏
符号表实现.替代的,我们提供了一个函数install(),它接受一个name对象并将其插入符号表
中。下面是符号表接口文件Name.h:
\begin{lstlisting}
extern void * symbol;   /* -> last Name found by screen() */
void install (const void * symbol);
int screen (const char * name);
\end{lstlisting}

    识别器必须将关键字如let插入符号表中,在筛选器寻找它们之前。这些关键字被定义在结构
体的常量表中--它们没有差别用于\ccode{install()}。以下函数用于初始化识别:
\begin{lstlisting}
#include "Name.h"
#include "Name.r"

static void initNames (void)
{
    static const struct Name names [] = {
        { 0, "let", LET },
        { 0 }};
    const struct Name * np;
    for (np = names; np -> name; ++np)
        install(np);
}
\end{lstlisting}

    注意\ccode{names[]},关键字表,不需要排序。要定义\ccode{names[]},我们使用
名称的表示形式,即,我们包括Name.r。因为关键字let被丢弃,所以我们没有提供动态链接的方法。

% \section{Superclass Implementation ---\cemph{Name}}
\section{父类的实现:\cemph{Name}}
    按名称搜索符号是一个基本问题。不幸的是,ANSI标准没有定义一个合适的库函数来
解决这个问题。\ccode{bsearch()}--排序表中的二进制搜索--接近了,但是如果我们插入
一个新的符号,我们就必须调用\ccode{qsord()}来为进一步的搜索做好准备。
    
    Unix系统可能提供两个或三个功能族来处理不断增长的表。\ccode{lsearch()}--对数组
进行线性搜索并在末尾(!)添加--并不完全有效。\ccode{hsearch()}--一个包含文本和信息
指针的结构的哈希表--只维护一个固定大小的表,并且在entrires上强加了一个笨拙的结构。
\ccode{tsearch()}--一种具有任意比较和删除的二叉树--是最普遍的家族,但是如果初始符号
是从排序序列中安装的。

    在UNIX系统上,\ccode{tsearch()}可能是最好的折衷方案。带有二进制线程树的可移植
实现的源代码可以在[Sch87]中找到。然而,如果这个家族不可用,或者如果我们不能保证随机
初始化,我们应该寻找一个更简单的工具来实现。事实证明,可以很容易地扩展bsearch()的
仔细实现,以支持插入到排序数组中:
\begin{lstlisting}
void * binary (const void * key, 
    void * _base, size_t * nelp, size_t width, 
    int (* cmp) (const void * key, const void * elt)) 
{   size_t nel = * nelp; 
#define base (* (char **) & _base) 
    char * lim = base + nel * width, * high; 
    
    if (nel > 0) 
    {   for (high = lim — width; base <= high; nel >>= 1) 
        {   char * mid = base + (nel >> 1) * width; 
            int c = cmp(key, mid); 
            
            if (c < 0) 
                high = mid — width; 
            else if (c > 0) 
                base = mid + width, —— nel; 
            else 
                return (void *) mid; 
        }
\end{lstlisting}

    到目前为止,这是在任意数组中的标准二进制搜索。key要查找对象的关键点;base基础是
*nelp元素表的起始地址,每一个都有width宽度字节;cmp是一个比较函数,用于将键与表元素
进行比较。此时,我们已经找到了一个表元素并返回了它的地址,或者基现在是键应该在表中的地址。
我们继续这样做:

\begin{lstlisting}
        memmove(base + width, base, lim — base);
    } 
    ++ *nelp; 
    return memcpy(base, key, width); 
#undef base 
} 
\end{lstlisting}

    \ccode{memmove()}将数组的末尾移开*,\ccode{memcpy()}插入键。我们假设在数组之外还有空间,
并通过nelp记录,我们添加了一个元素-\ccode{binary()}与标准函数\ccode{bsearch()}不同,只需要
地址,而不是包含表中元素数的变量的值。

    考虑到搜索和录入的一般方法,我们可以很容易地管理我们的符号表。
首先,我们需要将一个关键字与一个表元素进行比较:

\begin{lstlisting}
static int cmp (const void * _key, const void * _elt) 
{   const char * const * key = _key; 
    const struct Name * const * elt = _elt; 
    return strcmp(* key, (* elt) —> name); 
} 
\end{lstlisting}

    作为键,我们只传递指向输入符号文本的指针的地址。当然,Name结构体,我们只查看它们的
.name组件。

    搜索或输入是通过调用具有适当参数的\ccode{binary()}来完成的。由于我们事先不知道符号的数量,
所以我们确保表始终有展开的空间:

\begin{lstlisting}
static struct Name ** search (const char ** name) 
{   static const struct Name ** names; /* dynamic table */ 
    static size_t used, max; 
    
    if (used >= max) 
    {   names = names 
            ? realloc(names, (max *= 2) * sizeof * names) 
            : malloc((max = NAMES) * sizeof * names); 
        assert(names); 
    } 
    return binary(name, names, & used, sizeof * names, cmp); 
} 
\end{lstlisting}

    NAMES是一个定义的常量,表条目的初始分配是表项的初始分配;每次我们用完时,我们都会将表的
大小翻一番。

    \ccode{search()}以一个指向文本的地址被发现并返回该表项的地址。如果文本不能在表中找到,
\ccode{binary()}已在表中插入键,即仅指向文本的指针,而不是一个结构名。这一战略是为\ccode{screen()}
效益,只有建立一个新表元素如果标识符从输入是未知的:
\begin{lstlisting}
int screen (const char * name) 
{   struct Name ** pp = search(& name); 
    
    if (* pp == (void *) name) /* entered name */ 
    * pp = new(Var, name); 
    symbol = * pp; 
    return (* pp) —> token; 
} 
\end{lstlisting}

    \ccode{screen()}让\ccode{screen()}查找要筛选的输入符号。如果指向符号文本的
指针被输入到符号表中,我们需要用描述新标识符的条目替换它。

    对于\ccode{screen()},新的标识符必须是变量。我们假设有一个类型描述Var,它
知道如何构造描述变量的名称结构,然后让\ccode{new()}来完成其余的工作。在任何情
况下,我们让符号指向符号表条目,然后返回它的.token值。

\begin{lstlisting}
void install (const void * np) 
{   const char * name = ((struct Name *) np) —> name; 
    struct Name ** pp = search(& name); 
    
    if (* pp != (void *) name) 
        error("cannot install name twice: %s", name); 
    * pp = (struct Name *) np; 
}
\end{lstlisting}

    \ccode{install()}稍微简单一些。我们接受一个name对象,让\ccode{search()}在符号表
中找到它。\ccode{install()}应该只处理新的符号,所以我们应该总是能够输入对象而不是它的名称。
否则,如果\ccode{search()}真的找到了一个符号,我们就有麻烦了。

% \section{Subclass Implementation ---\cemph{Var}}
\section{子类的实现:\cemph{Var}}

    \ccode{screen()} 调用 \ccode{new()} 来创建一个新的变量符号表,并将其返回给识别器,该
识别器将其出入表达树中.因此,VAR必须创建可以向节点那样工作的符号表条目,也就是说,在定义
\\ccode{struct Var}时,我们需要扩展一个struct名称,已集成驻留在符号表中的能力,并且必须支持
适用于表达式接待的动态连接函数.我们在Var.h中描述了接口:

\begin{lstlisting}
const void * var;
const void * Assign;
\end{lstlisting}

    变量具有名称和值。如果我们评估算术表达式,我们需要返回.Value组件。
如果删除表达式,则不能删除变量节点,因为它位于符号表中:

\begin{lstlisting}
struct Var {    struct Name _; double value; };
#define value(tree) (((struct Var *) tree) -> value)
static double doVar (const void * tree)
{
    return value(tree);
}
static void freeVar (void * tree){}
\end{lstlisting}

正如在4.6节中所讨论的那样,通过为值提供访问函数来简化代码。

创建变量需要分配一个\ccode{struct Var},插入变量名的动态副本,以及识别器规定的令牌值VAR:

\begin{lstlisting}
static void * mkVar (va_list ap)
{   struct Var * node = calloc(1, sizeof(struct Var));
    const char * name = va_arg(ap, const char *);
    size_t len = strlen(name);
    
    assert(node);
    node -> _.name = malloc(len + 1);
    assert(node -> _.name);
    strcpy((void *) node -> _.name, name);
    node -> _.token = VAR;
    return node;
}
static struct Type _Var = { mkVar, doVar, freeVar };
const void * Var = &_Var;
\end{lstlisting}

    \ccode{new()}负责在符号返回到\ccode{screen()}或任何想使用它的人之前将类型描述Var
插入到节点中。

    从技术上讲,\ccode{mkVar()}是Name的构造函数。但是,只需要动态地存储变量名称。
因为我们决定在计算器中构造函数负责分配对象,所以我们不能让Var构造函数调用名称构造函数
来维护.name和.Token组件--Name构造函数将分配一个\ccode{struct Name},而不是\ccode{struct Var}。


% \section{Assignment}
\section{赋值}
\begin{lstlisting}
\end{lstlisting}

% \section{Anothor Subclass ---\cemph{Constants}}
\section{另一个子类:\cemph{Constants}}<++>

% \section{Mathematical Functions ---\cemph{Math}}<++>
\section{数学函数:\cemph{Math}}<++>

% \section{Summary}
\section{小结}<++>

% \section{Exercises}
\section{练习}<++>

A quick brown fox jumps over the lazy dogs.
A quick brown fox jumps over the lazy dogs.
A quick brown fox jumps over the lazy dogs.

A quick brown fox jumps over the lazy dogs.
A quick brown fox jumps over the lazy dogs.
A quick brown fox jumps over the lazy dogs.

\newpage{\thispagestyle{empty}\cleardoublepage}
% vim: set syntax=tex ts=4 sw=4 tw=76 fo+=Mm cc=+2 noundofile nobackup :

