% # Copyright (C) 2009-2014 the Fandol Team, All wrongs reserved.
% # -*- coding: utf-8 -*-
% !TEX encoding = UTF-8 Unicode

% Version Control System Information: Subversion, host on Google Code;
% FileID:		$Id$;
% FileDate:		$Date$;
% FileRevision:	$Revision$

% \chapter{Inheritance ---Code Reuse and Refinement}
\chapter{继承:代码重用和精炼}
\label{ch:Inheritance}

% \section{A Superclass ---\cemph{Point}}
\section{A Superclass:\cemph{Point}}

我们在这个章节里将开始一个基本绘画程序.
我们应该有一个像这样的测试类(class)的代码:

\begin{lstlisting}
#include "Point.h"
#include "new.h"

int main(int argc, char ** argv)
{
    void *p;
    
    while (* ++ argv)
    {
        switch (** argv)
        {
            case 'p':
                p = new(Point, 1, 2);
                break;
            default:
                continue;
        }
        draw(p);
        move(p, 10, 20);
        draw(p);
        delete(p);
    }
    return 0;
}
\end{lstlisting}


对于每一个命令参数以字符 p 开始,我们获得一个新的绘图的点,
移动这个点到某处,从新绘制,并且删除。
标准化C语言不包含图形化输出标准的函数:
然而,如果我们坚持产生一幅图片,
我们能够发表文本,对于这个文本Kernighan 的图片 [Ker82] 能够理解:

\begin{lstlisting}
$ points p
"." at 1,2
"." at 11,22
\end{lstlisting}

坐标对于测试是无关紧要的——从商业和面向对象的说法解释:“点就是一则消息。”


我们用这个点能做些什么呢?new() 将产生一个点,
并且构造器期望着初始化坐标作为进一步的参数传进 new() 。
通常,delete() 将回收我们的点并且按照惯例调用析构器。


draw() 安排点被显示出来。
由于我们希望与其他图形对象协同工作——因此在测试程序中会有switch——对于draw() 
我们将提供动态连接。


move() 通过传递一系列参数来改变点的坐标。
如果我们实现每一个图形对象,
这些对象都与它涉及的点关联,
我们将能够通过简单的应用这个点的move() 方法来移动它。
因此,对于move() 在不需要动态连接的情况下我们应该可以做。

% \section{Superclass Implementation ---\cemph{Point}}
\section{Superclass Implementation:\cemph{Point}}
在抽象数据类接口Point.h中有:

\begin{lstlisting}
extern const void * Point;			/* new(Point, x, y) */
void move(void * point, int dx, int dy);
\end{lstlisting}

我们重复利用第二章new.?文件,我们移除new中一下方法并且增添draw()到 new.h中:
\begin{lstlisting}
void * new (const void * class, ...);
void delete(void * item);
void draw(const void * self);
\end{lstlisting}

在 new.r 中声明 \ccode{struct Class} 应该相关联于new.h 





% \section{Inheritance ---\cemph{Circle}}
\section{继承:\cemph{Circle}}<++>

% \section{Linkage and Inheritance}
\section{联系与继承}<++>

% \section{Static and Dynamic Linkage}
\section{静态和动态链接}<++>

% \section{Visibility and Access Functions}
\section{Visibility and Access Functions}<++>

% \section{Subclass Implementation ---\cemph{Circle}}
\section{父类的实现:\cemph{Circle}}<++>

% \section{Summary}
\section{小结}<++>

% \section{Is It or Has It? ---Inheritance VS. Aggregates}
\section{Is It or Has It? ---Inheritance VS. Aggregates}<++>

% \section{Multiple Inheritance}
\section{多继承}<++>

% \section{Exercises}
\section{练习}<++>

A quick brown fox jumps over the lazy dogs.
A quick brown fox jumps over the lazy dogs.
A quick brown fox jumps over the lazy dogs.

A quick brown fox jumps over the lazy dogs.
A quick brown fox jumps over the lazy dogs.
A quick brown fox jumps over the lazy dogs.

\newpage{\thispagestyle{empty}\cleardoublepage}
% vim: set syntax=tex ts=4 sw=4 tw=76 fo+=Mm cc=+2 noundofile nobackup :

