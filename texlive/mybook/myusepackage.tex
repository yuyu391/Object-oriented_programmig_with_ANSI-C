%% STSong Font:
\usepackage{xeCJK}

% \setCJKmainfont{} % 设置 CJK 主字体,也就是设置 \rmfamily 的 CJK 字体
% \setCJKsansfont{} % 设置 CJK 无衬线的字体,也就是设置 \sffamily 的 CJK 字体
% \setCJKmonofont{} % 设置 CJK 的等宽字体,也就是设置 \ttfamily 的 CJK 字体

\setCJKmainfont[BoldFont=STZhongsong, ItalicFont=STFangsong]{STSong} %BoldFont 粗体字 % ItalicFont 斜体字
\setCJKsansfont[BoldFont=SimHei, ItalicFont=LiSu]{STXihei} % SimXX
% \setCJKsansfont[BoldFont=STXinwei, ItalicFont=STXingkai]{STKaiti}
\setCJKmonofont{STSong} % SIMYOU常规幼圆字体


\usepackage{indentfirst} % 首行缩进
\setlength{\parindent}{2.0em} % 首行缩进两字符
% Maybe useless when using zhspacing, the default in zhspacing is 2em.

\usepackage[pagestyles]{titlesec} % 标题
\usepackage{titletoc}

%% C代码编辑格式part01
\usepackage{xcolor} % 颜色
\usepackage{framed} % 代码背景色
\usepackage{verbatim} % 代码
\definecolor{shadecolor}{gray}{0.85} % 代码背景色灰色

%% C代码编辑格式part02
\usepackage{listings}
\lstset{
language=[ANSI]C,
basicstyle=\ttfamily\small\linespread{1}\selectfont,
numbers=left,
numberstyle=\footnotesize,
numbersep=10pt,
numberfirstline=false,
numbersep=0.5em,
stepnumber=1,
breaklines=true,
extendedchars=false,
tabsize=4,
xleftmargin=2em,
% xrightmargin=0em,
% framexleftmargin=-10mm,
% framexrightmargin=-10mm,
% frame=none,
backgroundcolor=\color[RGB]{240,240,240},
keywordstyle=\bfseries\color{blue},
% identifierstyle=,
numberstyle=\color[RGB]{0,192,192},
commentstyle=\slshape\color[RGB]{0,96,96},
stringstyle=\color[RGB]{128,0,0},
% title=\lstname,
showstringspaces=false
}

%% 页边距设置
\usepackage[top=3.0cm,bottom=3.0cm,left=2.8cm,right=2.8cm]{geometry}